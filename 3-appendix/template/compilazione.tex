\chapter{Generare il file PDF della tesi}
\label{appendix:compilazione}

\acresetall
% Empty the memory of the "\ac" macro: each time you use an acronym for the first time, the full name and the short name in brackets will be printed; afterwards only the acronym will be printed.

\section{Introduzione all'appendice}

In questa appendice forniamo istruzioni dettagliate per generare un file PDF a partire dal documento \latex del modello di tesi. Illustreremo, inoltre, come definire un comando di compilazione rapida per l'editor \href{https://www.xm1math.net/texmaker/}{\textit{Texmaker}}.

\section{Generare il file PDF da linea di comando}

Per generare il file PDF della tesi, è sufficiente eseguire i seguenti comandi{\footnotemark} nella root directory della repository git:

\begin{lstlisting}[language=bash]
pdflatex -interaction=nonstopmode thesis-template.tex && \
bibtex thesis-template.aux && \
pdflatex -interaction=nonstopmode thesis-template.tex && \
pdflatex -interaction=nonstopmode thesis-template.tex
\end{lstlisting}

\footnotetext{I comandi qui presentati sono stati testati utilizzando pdf{\TeX} 3.14159265-2.6-1.40.18 ({\TeX} Live 2017/Debian).}

\medskip

Illustriamo brevemente il loro funzionamento:
\begin{itemize}

\item il primo \texttt{pdflatex} compila il documento \latex , producendo una versione iniziale del PDF, il file ausiliario (\texttt{.aux}) e diversi altri file intermedi \cite{latex2023manual}; il PDF prodotto non è completo: mancano l'indice, gli elenchi (di figure, tabelle e codici) e la bibliografia; inoltre, le citazioni e i riferimenti a capitoli, sezioni, tabelle, figure o codici non sono ancora stati risolti;

\item il secondo comando esegue Bib\TeX, il software utilizzato per gestire e formattare la bibliografia; questo analizzerà il file ausiliario generato dalla prima compilazione e i database bibliografici (\texttt{.bib}) per creare il file bibliografico \latex (\texttt{.bbl})\cite{bibtex2007faq};

\item le ultime due compilazioni con \texttt{pdflatex} sono necessarie per ottenere la versione finale del documento PDF \cite{latex2023manual};

\item l'opzione \texttt{-interaction=nonstopmode} consente a \latex di continuare la compilazione anche in presenza di errori o avvisi: eventuali messaggi diagnostici verranno mostrati sul terminale, ma non sarà richiesta alcuna interazione con l'utente \cite{latex2023manual}.

\end{itemize}

Per generare l'indice analitico non è stato eseguito alcun comando aggiuntivo: il pacchetto \texttt{imakeidx}, una volta incluso \texttt{\textbackslash makeindex} nel preambolo, consente la sua generazione in modo automatico durante la prima compilazione \cite{imakeidx2023doc}.

\medskip

Se al posto di \texttt{imakeidx} avessimo usato \texttt{makeidx}, sarebbe stato necessario eseguire come secondo comando \texttt{makeindex thesis-template.idx}: questo elabora l'indice analitico grezzo (\texttt{.idx}), prodotto dalla prima compilazione, per creare la sua versione finale (\texttt{.ind}) \cite{latex2023manual}.

\section{Compilazione rapida su Texmaker}

Se stiamo utilizzando l'editor \href{https://www.xm1math.net/texmaker/}{\textit{Texmaker}}, possiamo adattare quanto visto nella sezione precedente per definire il seguente comando di compilazione rapida:

\begin{lstlisting}[language=bash]
pdflatex -synctex=1 -interaction=nonstopmode %.tex|
bibtex %.aux|
pdflatex -synctex=1 -interaction=nonstopmode %.tex|
pdflatex -synctex=1 -interaction=nonstopmode %.tex|
evince %.pdf
\end{lstlisting}

\medskip

Evidenziamo le modifiche apportate:
\begin{itemize}

\item ai comandi \texttt{pdflatex} abbiamo aggiunto l'opzione \texttt{-synctex=1}; questa attiva \textit{SyncTeX}, una funzionalità che crea sincronizzazione tra il codice sorgente e il PDF generato, consentendo la navigazione bidirezionale tra i due \cite{texmaker2023manual};

\item al posto di \texttt{thesis-template}, usiamo il simbolo \texttt{\%}, un segnaposto che verrà automaticamente sostituito con il nome del file in elaborazione \cite{texmaker2023manual};

\item è stata aggiunta l'istruzione finale \texttt{evince \%.pdf} che apre il PDF generato in \textit{Evince} (un visualizzatore di PDF per Linux).

\end{itemize}

Se invece del pacchetto \texttt{imakeidx} usassimo \texttt{makeidx}, dovremmo modificare il comando di compilazione rapida aggiungo \texttt{makeindex \%.idx|} subito dopo il primo \texttt{pdflatex}.