\chapter{Template math tests}
\label{appendix:math-test}
\acresetall

\section{Appendix introduction}

In this appendix we will test some math related keywords, packages and their settings.

\section{Test math keywords}

\subsection{Sets of numbers}

\begin{enumerate}

\item Natural numbers: \N.
\begin{itemize}
\item \N in text mode.
\item $\N$ in math mode.
\end{itemize}

\item Integer numbers: \Z.
\begin{itemize}
\item \Z in text mode.
\item $\Z$ in math mode.
\end{itemize}

\item Rational numbers: \Q.
\begin{itemize}
\item \Q in text mode.
\item $\Q$ in math mode.
\end{itemize}

\item Real numbers: \R.
\begin{itemize}
\item \R in text mode.
\item $\R$ in math mode.
\end{itemize}

\item Complex numbers: \C.
\begin{itemize}
\item \C in text mode.
\item $\C$ in math mode.
\end{itemize}

\item In math mode "xspace" does not work:
\begin{itemize}
\item $n \in \N c \in \C$.
\end{itemize}

\end{enumerate}

\subsection{Custom spaced logical operators}

\begin{enumerate}
\item AND: $(a,b) \in V \myland (b,c) \in V$
\item OR: $(a,b) \in V \mylor (b,c) \in V$
\end{enumerate}

\subsection{Text conjunction keywords}

\begin{enumerate}
\item "e": $n \in \N \e c \in \C$.
\item "allora" : $n \geq 5 \allora n \geq 4$.
\item "se": $n \geq 5 \se n = 5$.
\item "per": $x_i \prec x_{i+1} \per 1 \leq i \leq n-1$.
\item "con": $x_i \prec x_{i+1} \ldots \con x_1 = s \text{,} \ x_n = s'$.
\item "implica":  $E_1 \rightarrow n \implica E_2 \rightarrow n$.
\end{enumerate}

\section{Test amsthm environments}

\subsection{Plain environments}

\subsubsection{Proposizione}

\begin{proposizione}[Proposizione di prova]
\lipsum[4]
\end{proposizione}

\begin{proposizione-num}{3.2.1}[Custom proposizione di prova]
\lipsum[4]
\end{proposizione-num}

\subsubsection{Lemma}

\begin{lemma}[Lemma di prova]
\lipsum[4]
\end{lemma}

\begin{lemma-num}{3.2.1}[Custom lemma di prova]
\lipsum[4]
\end{lemma-num}

\subsubsection{Teorema}

\begin{teorema}[Teorema di prova]
\lipsum[4]
\end{teorema}

\begin{teorema-num}{3.2.1}[Custom teorema di prova]
\lipsum[4]
\end{teorema-num}

\subsubsection{Corollario}

\begin{corollario}[Corollario di prova]
\lipsum[4]
\end{corollario}

\begin{corollario-num}{3.2.1}[Custom corollario di prova]
\lipsum[4]
\end{corollario-num}
 
\subsubsection{Congettura}

\begin{congettura}[Congettura di prova]
\lipsum[4]
\end{congettura}

\begin{congettura-num}{3.2.1}[Custom congettura di prova]
\lipsum[4]
\end{congettura-num}

\subsection{Definition environments} 

\subsubsection{Definizione}

\begin{definizione}[Definizione di prova]
\lipsum[4]
\end{definizione}

\begin{definizione-num}{3.2.1}[Custom definizione di prova]
\lipsum[4]
\end{definizione-num}

\subsubsection{Assioma}

\begin{assioma}[Assioma di prova]
\lipsum[4]
\end{assioma}

\begin{assioma-num}{3.2.1}[Custom assioma di prova]
\lipsum[4]
\end{assioma-num}

\subsubsection{Problema}

\begin{problema}[Problema di prova]
\lipsum[4]
\end{problema}

\begin{problema-num}{3.2.1}[Custom problema di prova]
\lipsum[4]
\end{problema-num}

\subsubsection{Esempio}

\begin{esempio}[Esempio di prova]
\lipsum[4]
\end{esempio}

\begin{esempio-num}{3.2.1}[Custom esempio di prova]
\lipsum[4]
\end{esempio-num}

\subsubsection{Esercizio}

\begin{esercizio}[Esercizio di prova]
\lipsum[4]
\end{esercizio}

\begin{esercizio-num}{3.2.1}[Custom esercizio di prova]
\lipsum[4]
\end{esercizio-num}

\subsubsection{Algoritmo}

\begin{algoritmo}[Algoritmo di prova]
\lipsum[4]
\end{algoritmo}

\begin{algoritmo-num}{3.2.1}[Custom algoritmo di prova]
\lipsum[4]
\end{algoritmo-num}

\subsection{Remark environments}

\subsubsection{Nota}

\begin{nota}[Nota di prova]
\lipsum[4]
\end{nota}

\begin{nota-num}{3.2.1}[Custom nota di prova]
\lipsum[4]
\end{nota-num}

\section{Test finite state machine}

In \cref{figure:fsm} we see an example of a finite state machine.

\begin{figure}[h]
\centering
\begin{tikzpicture}
[initial text = {}, every initial by arrow/.style={-Stealth, thick}, node distance = 3cm]
\node[elliptic state, initial](q0){$a;(b+c)$};
\node[elliptic state, right of=q0](q1){$1;(b+c)$};
\node[elliptic state, right of=q1](q2){$b+c$};
\node[elliptic state, right of=q2, accepting](q3){$1$};
\path[-Stealth, thick, above, every loop/.style={-Stealth}]
	(q0) edge node {$a$} (q1)
	(q1) edge node {$\varepsilon$} (q2)
	(q2) edge[bend left] node {$b$} (q3)
		 edge[bend right] node {$c$} (q3)
	(q3) edge[loop right] node {$\varepsilon$} (q3);
\end{tikzpicture}
\caption[FSM example]{just a finite state machine example.}
\label{figure:fsm}
\end{figure}