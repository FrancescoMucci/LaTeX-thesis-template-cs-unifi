\chapter{Consigli per la scrittura della tesi}
\label{appendix:scrittura}

\acresetall
% Empty the memory of the "\ac" macro: each time you use an acronym for the first time, the full name and the short name in brackets will be printed; afterwards only the acronym will be printed.

\section{Introduzione all'appendice}

In questa appendice riportiamo alcune consigli utili per la stesura della tesi; in particolare, rispondiamo alle seguenti domande: 
\begin{itemize}
\item Qual è il vero scopo di una tesi?
\item Quali capitoli dovrebbe includere?
\item Quanto dovrebbe essere lunga?
\item Quante pagine dedicare ad ogni capitolo?
\item Quale approccio adottare per la sua scrittura?
\item In quale ordine scriviamo i capitoli?
\end{itemize}

\section{Scopo della tesi}

Prima di iniziare a lavorare alla tesi, è fondamentale comprendere che il suo scopo principale è evidenziare le competenze dal candidato. Questa dovrà mostrare che lo studente padroneggia i fondamenti della disciplina, è in grado di svolgere un lavoro di ricerca in modo indipendente, sa interpretare correttamente i risultati ottenuti e ha ottime capacità comunicative \cite{zobel2015writing}.

\medskip

Pertanto, il nostro lavoro sarà valutato non tanto per la bontà dell'idea presentata (di cui avremo ampiamente discusso con il relatore), ma piuttosto per la qualità della scrittura e la capacità di presentazione mostrata durante la discussione \cite{pfandzelter2022thesis}.

\medskip

Anche una tesi che non fornisce un contributo di ricerca significativo può mostrare le abilità del candidato, in particolare se il pensiero critico dello studente si manifesta chiaramente. Un modo eccellente per mostrare questa competenza è attraverso un'accurata analisi critica del lavoro svolto (da includere normalmente nella sezione conclusiva della tesi), che permetta agli esaminatori di valutare la reale comprensione che il candidato ha dei metodi usati e dei risultati ottenuti \cite{zobel2015writing}.

\section{Capitoli della tesi}

Seguendo l'approccio proposto da Pfandzelter \etAl \cite{pfandzelter2022thesis}, una tesi tecnico{\hyp}scientifica dovrebbe essere costituita dai seguenti capitoli:

\begin{enumerate}

\item \textit{Introduzione}: presenta il problema di ricerca trattato, illustra gli obiettivi del lavoro e delinea una panoramica del suo contenuto;

\item \textit{Preliminari}: introduce le nozioni fondamentali per la comprensione del lavoro svolto e inquadra la tesi collegandola a eventuali lavori precedenti;

\item \textit{Approccio}: illustra l'idea risolutiva proposta per il problema affrontato;

\item \textit{Valutazione}: dimostra, attraverso un processo di valutazione, l'efficacia dell'approccio risolutivo adottato;

\item \textit{Discussione}: conduce un'analisi critica e oggettiva dell'approccio e del metodo di valutazione utilizzati;

\item \textit{Lavori correlati}: esamina le ricerche correlate mettendole a confronto con il lavoro realizzato;

\item \textit{Conclusioni}: riassume in modo conciso quanto svolto (concentrandosi principalmente sui risultati ottenuti) e suggerisce possibili sviluppi futuri.

\end{enumerate}

\section{Lunghezza della tesi}

Per pianificare accuratamente il lavoro di scrittura, è fondamentale avere un'idea chiara della lunghezza desiderata per la nostra tesi e del numero di pagine da destinare ad ogni suo capitolo \cite{mannisto2022guide}. La quantità di pagine assegnate sarà indicativa dell'importanza dello specifico capitolo \cite{tuni2019guide}.

\medskip

Solitamente, la lunghezza di una tesi di laurea varia tra le 18 e le 28 pagine per una triennale e tra le 55 e le 75 per una magistrale; le tesi di dottorato, invece, presentano una lunghezza più variabile, che può andare da un minimo di 30 pagine fino ad un massimo di 200 \cite{tuni2019guide}.

\section{Lunghezza dei capitoli}

Tenendo presente di quanto detto nella sezione precedente ed assumendo di lavorare ad una tesi magistrale, possiamo suddividere le pagine tra i vari capitoli nel modo seguente: per l'\textit{Introduzione}, prevediamo 3-5 pagine; per i \textit{Preliminari}, l'\textit{Approccio} e la \textit{Valutazione}, ne assegniamo 10-15 ciascuno; per la \textit{Discussione} e i \textit{Lavori correlati}, 5-7 ognuno; e infine, ne destiniamo 2-4 per le \textit{Conclusioni}. Aggiungendo fino a 7 pagine per eventuali appendici e 5-10 per la bibliografia, dovremmo essere in grado di produrre una tesi con un totale di pagine comprese tra un minimo di 50 ed un massimo di 85.

\section{Metodologia di scrittura}

In questa sezione riassumiamo la metodologia di scrittura proposto dalla \textit{Tampere University} \cite{tuni2019guide}.

\subsection{Studio preparatorio}

Dopo aver definito l'argomento della tesi con il relatore, il primo passo nel processo di scrittura è lo studio preparatorio: ciò implica la lettura della letteratura esistente (lavori precedenti e correlati) e la raccolta di materiale per i preliminari. Questa fase iniziale è essenziale per acquisire una preparazione adeguata nel proprio campo di ricerca.

\medskip

La revisione della letteratura deve essere un'attività proattiva: non si tratta solo di leggere gli articoli, ma anche di iniziare a scrivere note che formeranno le fondamenta della tesi\footnotemark. È importante riassumere sinteticamente le letture e annotare osservazioni e riflessioni rilevanti per il nostro lavoro. La produzione di queste note ci aiuterà a chiarire i pensieri, stimolare l'apprendimento ed elaborare nuove idee.

\footnotetext{Le note prodotte dovrebbero includere anche i riferimenti bibliografici corretti: recuperarli successivamente non è pratico.}

\subsection{Definizione dell'indice}

Dopo aver completato lo studio preparatorio, possiamo definire l'indice effettivo della tesi a partire da quello definito in questo modello. Per ogni capitolo dobbiamo stabilire un titolo appropriato, riassumerne il contenuto e rivedere il numero di pagine pianificate. Possiamo anche rinominare le varie sottosezioni e fornire per ciascuna una breve descrizione del contenuto. Questa fase di strutturazione dell'indice ci aiuterà a ridurre il carico decisionale e a concentrarci meglio sulla scrittura della tesi.

\subsection{Processo iterativo di scrittura}

L'intero processo di scrittura è iterativo: a partire dalle note realizzate durante lo studio preliminare, svilupperemo bozze dei vari capitoli che verranno continuamente riviste e rielaborate durante il lavoro, fino al raggiungimento della versione finale. Seguendo questo metodo, al momento della scrittura della prima bozza di ogni capitolo, potremo concentrarci principalmente sul contenuto piuttosto che sulla forma.

\medskip

Una volta completata la stesura delle bozze, ci dedicheremo alla revisione del contenuto e alla correzione della forma: i capitoli che in questa fase richiedono maggior lavoro sono, normalmente, le \textit{Conclusioni}, l'\textit{Introduzione} e i \textit{Preliminari}. 

\subsection{Rilettura finale}

Infine, effettueremo una rilettura approfondita della tesi per garantire l'assenza di errori e per assicurarci che la trattazione sia fluida e coerente. È anche consigliabile farla rileggere ad amici o parenti per avere ulteriori riscontri.

\section{Ordine di scrittura dei capitoli}

Una volta determinato il numero di pagine per ogni capitolo e chiarita la metodologia di scrittura da adottare, il passo successivo è stabilire l'ordine di scrittura dei capitoli.

\subsection{Primi capitoli}

Come discusso nella sezione precedente, partiremo avendo delle note iniziali riguardo i lavori precedenti, i lavori correlati e alcune nozioni preliminari. Queste note costituiranno le bozze per i capitoli \textit{Preliminari} e \textit{Lavori correlati}; tuttavia, non completeremo subito la loro scrittura onde evitare di includere più informazioni del necessario (non vogliamo trasformare la tesi in un libro di testo).

\medskip

Potremmo, dunque, iniziare con i due capitoli centrali, \textit{Approccio} e \textit{Valutazione}; in tal modo, cominceremmo fin da subito a lavorare alla risoluzione del problema di ricerca affrontato. 

\medskip

%Se stessimo realizzando un progetto implementativo, questa scelta farebbe sì che, dopo lo studio iniziale, ci concentrassimo prima sulla specifica e sul design del software e poi sulla sua implementazione e verifica (formale o tramite test).

Se stessimo realizzando un progetto implementativo, questa scelta fa sì che, dopo lo studio iniziale, ci concentreremo prima sulla specifica e sul design del software e poi sulla sua implementazione e verifica (formale o tramite test).

\medskip

Parallelamente alla scrittura dei primi capitoli, terremo aggiornate le bozze dei \textit{Preliminari} e dei \textit{Lavori correlati} e, eventualmente, inizieremo ad annotare anche gli altri.

\subsection{Capitoli successivi}

Dopo aver ideato e valutato il nostro approccio risolutivo, abbiamo tutti quello che ci serve per completare i \textit{Lavori correlati}: saremo in grado, infatti, di confrontare la nostra idea con le ricerche analizzate, evidenziando le influenze ricevute e le lacune colmate.

\medskip

Con l'analisi dei lavori correlati conclusa, possediamo tutti i dati necessari per una valutazione critica e obiettiva del nostro lavoro; possiamo, quindi, procedere con la \textit{Discussione}.

\medskip

A questo punto abbiamo anche una chiara comprensione delle nozioni preliminari necessarie per il nostro lavoro e siamo in grado, di conseguenza, di completare i \textit{Preliminari} aggiungendo contenuti necessari o eliminando quelli superflui.

\subsection{Ultimi capitoli}

Una volta completata la maggior parte della tesi, possiamo riassumere efficacemente il lavoro svolto e riflettere sui possibili sviluppi futuri: scriviamo, quindi, le \textit{Conclusioni}.

\medskip

Infine, ci occupiamo dell'\textit{Introduzione}: avendo una visione completa dell'intero lavoro, possiamo presentare con chiarezza il problema di ricerca trattato, gli obiettivi della tesi e la panoramica del suo contenuto.

\section{Ulteriori consigli}

Riportiamo, infine, alcuni consigli addizionali forniti dalla \textit{Tampere University} \cite{tuni2019guide}.

\begin{itemize}

\item Imponiti una schedule con milestone chiare che siano sia ambiziose che realistiche.

\item Evita di pianificare eccessivamente e di procrastinare.

\item Stabilisci una routine di scrittura produttiva (impegnati a scrivere qualcosa ogni giorno).

\item Se riesci a scrivere 1-3 pagine di buona qualità al giorno, stai facendo degli ottimi progressi.

\item Non scrivere in solitudine, consulta regolarmente il tuo relatore.

\item Non avere paura del fallimento (non esitare a mostrare il tuo lavoro non finito).

\item Essere perfezionisti è un ostacolo alla produttività.

\item L'impegno e un atteggiamento positivo sono le chiavi per superare qualsiasi ostacolo.

\end{itemize}