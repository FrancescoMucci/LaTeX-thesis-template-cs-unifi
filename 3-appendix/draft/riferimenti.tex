\addChapterBookmark{Riferimenti bibliografici per argomento}{bookmark:bibref}
% Although we do not include an entry for the references list in the table of contents itself, we create a bookmark for it at level 0.
\chapter*{Riferimenti bibliografici per argomento}
\label{appendix:bibref}
% The starred variants of all sectioning commands (\chapter*, \section*, ...) produce unnumbered headings which do not appear in the table of contents or in the page header. The absence of a running head often has an unwanted side effect: if, for example, a chapter set using \chapter* spans several pages, then the running head of the previous chapter suddenly reappears; so we need to use the command "\markboth{<left-mark>}{<right-mark>}" to customize the page header (the left-mark will normally be used in the header of even pages and right-mark in the header of odd pages; with one-sided printing, only the right-mark exists).
\markboth{Riferimenti bibliografici per argomento}{Riferimenti bibliografici per argomento}

\acresetall
% Empty the memory of the "\ac" macro: each time you use an acronym for the first time, the full name and the short name in brackets will be printed; afterwards only the acronym will be printed.

%----------SUGGERIMENTI-SUL-CONTENUTO----------%
%Le appendici dovrebbero includere materiale che, pur essendo necessario per comprendere il lavoro svolto, non rappresenta una parte centrale della tesi o non può essere inserito nel testo principale a causa delle sue dimensioni eccessive o del formato particolare \cite{tuni2019guide}.
%----------------------------------------------%

\addSectionBookmark{Introduzione all'appendice}{bookmark:bibref-intro}
\section*{Introduzione all'appendice}

%----------SUGGERIMENTI-SUL-CONTENUTO----------%
%Poiché è possibile che il lettore consulti le appendici senza aver letto integralmente la tesi, è consigliabile includere in ognuna di queste una breve introduzione che ne descriva il contenuto e che la collochi nel contesto più ampio del lavoro svolto \cite{tuni2019guide}.
%----------------------------------------------%

In questa appendice, riservata unicamente alla bozza della tesi, vengono presentati i riferimenti bibliografici consultati, organizzati in base all'argomento e alla tipologia di documento.

\addSectionBookmark{Argomento x}{bookmark:bibref-x}
\section*{Argomento x}

\subsection*{Libri di testo}
\begin{itemize}

\item \cite{};

\item \cite{}.

\end{itemize}

\subsection*{Capitoli di libri}
\begin{itemize}

\item \cite{};

\item \cite{}.

\end{itemize}

\subsection*{Revisioni sistematiche}
\begin{itemize}

\item \cite{};

\item \cite{}.

\end{itemize}

\subsection*{Lavori seminali}
\begin{itemize}

\item \cite{};

\item \cite{}.

\end{itemize}

\subsection*{Articoli scientifici}
\begin{itemize}

\item \cite{};

\item \cite{}.

\end{itemize}

\subsection*{Documentazione}
\begin{itemize}

\item \cite{};

\item \cite{}.

\end{itemize}

\subsection*{Tesi di dottorato}
\begin{itemize}

\item \cite{};

\item \cite{}.

\end{itemize}