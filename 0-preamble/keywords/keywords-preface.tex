%--------------------------------------------------------------
%                 Keywords for thesis body
%--------------------------------------------------------------

%----------------------KEYWORDS--------------------------------

% The "\keyword{\<name>}{<text>}" command can be used to define a simple keyword: useful to never forget the "\xspace" at the end of the definition (without the "\xspace" every keywords inserted inside text should be followed or enclosed by curly brackets or else there would be no space added).

\keyword{\latex}{\LaTeX}

%--------------------------------------------------------------

%----------------------EMPHASIZED-KEYWORDS---------------------

% The "\emphKeyword{\<name>}{<text>}" command can be used to define a keyword with emphasized text.

\emphKeyword{\myAntiplagio}{\href{https://unifi.turnitin.com/home}{Turnitin}}

%--------------------------------------------------------------

%----------------------TEXTTT-KEYWORDS-------------------------

% The "\ttKeyword{\<name>}{<text>}" command can be used to define a keyword with typewriter font (useful for code-related keywords).

%--------------------------------------------------------------

%----------------------ACRONYM-KEYWORDS------------------------

% The "\acroKeyword{\<name>}{<acronym>}" command can be used to define a keywords for a defined acronym.

%--------------------------------------------------------------

%----------------------INDEXED-KEYWORDS------------------------

% The "\idxKeyword{\<name>}{<text>}{<entry>[!<subentry>]}" command can be used to define a indexed keyword: every time it is used, it adds the specified entry in the analytical index.

%--------------------------------------------------------------

%----------------------INDEXED-EMPHASIZED-KEYWORDS-------------

% The "\idxEmphKeyword{\<name>}{<text>}{<entry>[!<subentry>]}" command can be used to define a indexed emphasized keyword.

%--------------------------------------------------------------

%----------------------INDEXED-TEXTTT-KEYWORDS-----------------

% The "\idxTtKeyword{\<name>}{<text>}{<entry>[!<subentry>]}" command can be used to define a indexed keyword with typewriter font (useful for code-related keywords).

%--------------------------------------------------------------

%----------------------INDEXED-ACRONYM-KEYWORDS----------------

% The "\idxAcroKeyword{\<name>}{<acronym>}{<entry>[!<subentry>]}" command can be used to define an indexed keyword for a defined acronym.

%--------------------------------------------------------------

%----------------------AUTO-INDEXED-ACRONYM-KEYWORDS-----------

% The "\autoIdxAcroKeyword{\<name>}{<acronym>}{<group>}" command can be used to define an indexed keyword for a defined acronym without the need to specify the entry to use in the analytical index: the entry will be in the specified group and will correspond to the short name associated with the acronym.

%--------------------------------------------------------------

%----------------------SECURE-WEBPAGE-KEYWORDS-----------------

% The "\webpage{\<name>}{<URL-without-protocol>}" command can be used to define a keyword for a "secure" webpage.

\webpage{\myThesisRepo}{github.com/FrancescoMucci/LaTeX-thesis-template-cs-unifi}

%--------------------------------------------------------------

%----------------------NOT-SECURE-WEBPAGE-KEYWORDS-------------

% The "\notsecurewebpage{\<name>}{<URL-without-protocol>}" command can be used to define a keyword for a "not secure" webpage.

%--------------------------------------------------------------

%----------------------MAIL-KEYWORDS---------------------------

% The "\mail{\<name>}{<email>}" command can be used to define a keyword for an email address.

%--------------------------------------------------------------