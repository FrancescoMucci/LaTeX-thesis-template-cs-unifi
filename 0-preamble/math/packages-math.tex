%---------------------------------------------------------------
%            Math related packages for thesis
%---------------------------------------------------------------

%--------------------USED-MATH-PACKAGES-------------------------

\usepackage[fleqn]{amsmath}
% The "amsmath" package is a LaTeX package that provides various extensions for improving the information structure and printing of documents containing mathematical formulas. The "fleqn" option position equations at a fixed indent from the left margin rather than centered in the text column.

\usepackage{amssymb}
% The "amssymb" package provides additional math symbols, like arrows, operators, special characters, geometric figures.

\usepackage{mathtools}
% The "mathtools" package is an extension package to amsmath. There are two things on "mathtools" agenda: (1) correct various bugs/deficiencies in amsmath until these are fixed by the AMS and (2) provide useful tools for mathematical typesetting for example the ability to write over arrows.

\usepackage{stmaryrd}
% The "stmaryrd" packages provides a number of new symbols for theoretical computer science, including ones for derivation of functional programming, process algebra, domain theory, linear logic, multisets and many more. It also fixes some features with AMS symbols and adds obvious variants of others.

\usepackage{amsthm}
% The "amsthm" package facilitates the kind of theorem setup typically needed in American Mathematical Society publications. This package provides an enhanced version of LaTeX’s "\newtheorem" command for defining theorem-like environments. The enhanced "\newtheorem" recognizes a "\theoremstyle" specification. The package also defines a proof environment that automatically adds a QED symbol at the end. If the "amsthm" package is used with a non-AMS document class and with the "amsmath" package, this must be loaded after "amsmath".

%---------------------------------------------------------------

%--------------------UNSED-MATH-PACKAGES------------------------

%\usepackage{mathpartir} 
% The "mathpartir" package provides macros for typesetting math formulas in mixed horizontal and vertical mode, e.g. fractions, inference rules and derivations.

%\usepackage{siunitx}
% The "siunitx" package handle all of the possible unit of measure related needs of LaTeX users.

%---------------------------------------------------------------

%--------------------SOURCES-FOR-COMMENTS-----------------------

% The comments on LaTeX and its commands are based on the contents of https://latexref.xyz/, an unofficial reference manual for the LaTeX2e document preparation system.

% The comments on the classes, styles or packages (and their commands and options) come from the description provided on CTAN (https://www.ctan.org/) and from the official documentation of the different classes, styles or packages.

%----------------------------------------------------------------