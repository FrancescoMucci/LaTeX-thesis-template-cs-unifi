\chapter{Approccio}
\label{chap:approccio}

\acresetall
% Empty the memory of the "\ac" macro: each time you use an acronym for the first time, the full name and the short name in brackets will be printed; afterwards only the acronym will be printed.

Questo è il primo dei due capitoli centrali in cui descriviamo il lavoro progettuale svolto. Il suo obiettivo principale è rispondere alla domanda: \textit{"Come abbiamo risolto il problema di ricerca affrontato?"} \cite{pfandzelter2022thesis}.

\medskip

Sarà qui che presenteremo, tramite una descrizione ad alto livello, la nostra idea risolutiva. A seconda della natura del problema affrontato, potremmo dunque trovarci a descrivere la progettazione di uno studio comparativo, l'architettura di un nuovo sistema, un nuovo algoritmo per la soluzione di un problema irrosolto oppure un algoritmo noto, ma che risolve un problema nuovo \cite{pfandzelter2022thesis}.

\medskip

Di norma, se la nostra domanda di ricerca è del tipo \textit{"Come posso risolvere questo problema?"}, la soluzione proposta dovrà essere presentata senza illustrare il processo iterativo che ci ha portato alla sua formulazione \cite{pfandzelter2022thesis}. La situazione cambia se la domanda di ricerca è \textit{"Quale tra i vari approcci noti è il migliore?"}: in questo caso, descriveremo come intendiamo eseguire il nostro studio comparativo, inserendo una descrizione dettagliata dei singoli approcci nel capitolo sulle nozioni preliminari \cite{pfandzelter2022thesis}.

\medskip

La struttura di questo capitolo sarà fortemente influenzata dal tipo di progetto intrapreso e generalmente coprirà le prime fasi del suo sviluppo \cite{unibz2022thesis}: per un progetto implementativo, includeremo la specifica dei requisiti e una descrizione ad alto livello del design del software, facendo uso di strumenti come pseudocodice o diagrammi di flusso per facilitarne la comprensione.

\section{Introduzione al capitolo}

All'inizio di ogni capitolo includeremo una breve introduzione che fornisce contesto al capitolo stesso; questa introduzione faciliterà la transizione logica da un capitolo all'altro mostrando come quello corrente si colleghi ai precedenti \cite{zobel2015writing}.

\medskip

Per essere più precisi, in ognuna di queste introduzioni forniremo \cite{unibz2022thesis}:
\begin{itemize}

\item gli obiettivi generali del capitolo, specificando cosa si intende affrontare e quale aspetto del nostro lavoro verrà esplorato;

\item una spiegazione di come il capitolo corrente si inserisca nel contesto più ampio della tesi, collegandolo ai temi generali e agli obiettivi più ampi del nostro lavoro;

\item un breve riassunto di come si è concluso il capitolo precedente e come quello corrente si costruisce sulle fondamenta del primo (ciò va fatto solo se pertinente);

\item un sommario del contenuto del capitolo, fornendo, ad esempio, una concisa panoramica delle sezioni e sottosezioni presenti.

\end{itemize}

\section{Specifica dei requisiti}

In questa sezione verranno elencati i requisiti funzionali e non funzionali del nostro progetto implementativo. Specificheremo, in sostanza, cosa il nostro sistema è tenuto a fare, ma non come lo andrà a fare \cite{shoaff2001thesis}.

\subsection{Requisiti funzionali}

I requisiti funzionali specificano le funzionalità che il sistema dovrà essere in grado di eseguire. Questi requisiti delineano gli input e gli output, le funzioni eseguite dal sistema e i dati che esso deve gestire. Sono inclusi anche i dettagli sulle interfacce utente e le interazioni tra il sistema e altri sistemi \cite{wah2009pedia}.

\subsection{Requisiti non funzionali}

I requisiti non funzionali descrivono le caratteristiche qualitative generali del sistema software. Questi comprendono aspetti come le prestazioni, l'usabilità, la sicurezza, l'affidabilità, la disponibilità, la manutenibilità e la portabilità. Includono anche i vincoli entro cui ci si aspetta che il sistema finale operi, quali il sistema operativo, la velocità di elaborazione, la larghezza di banda della rete, la capacità di memoria e il linguaggio di programmazione utilizzato \cite{wah2009pedia}.

\section{Architettura del sistema}

In questa sezione descriviamo l'architettura del sistema software che intendiamo implementare.

\subsection{Design architetturale del sistema}

Illustreremo, anzitutto, la struttura del nostro sistema, evidenziando le relazioni tra i vari sottosistemi che lo compongono. Ci concentreremo su una visione ad alto livello, evitando di addentrarci eccessivamente nei dettagli specifici;  per far ciò, forniremo una descrizione generale di come le responsabilità siano state suddivise e assegnate alle diverse componenti e di come queste interagiscono tra loro per realizzare le funzionalità desiderate. Al fine di rendere la trattazione più chiara, includeremo una rappresentazione visiva dell'architettura del sistema, utilizzando, ad esempio, un diagramma UML \cite{harran2023design}.

\subsection{Componente 1 del sistema}

Per ogni componente del sistema, forniremo un'analisi dettagliata delle sue responsabilità e delle sue interfacce di input e di output. Presenteremo, ove necessario, una descrizione dei suoi aspetti algoritmici e, inoltre, analizzeremo il modo in cui essa interagisce con le altre componenti, utilizzando ad esempio dei sequence diagram per illustrare ciascun caso d'uso \cite{harran2023design}.

\subsubsection{Responsabilità}

\subsubsection{Interfacce}

\subsubsection{Dettagli algoritmici}

\subsubsection{Comportamento dinamico}

\subsection{Componente n del sistema}

\subsubsection{Responsabilità}

\subsubsection{Interfacce}

\subsubsection{Dettagli algoritmici}

\subsubsection{Comportamento dinamico}

\subsection{Considerazioni sulle scelte architetturali}

In questa sezione spieghiamo le ragioni che hanno guidato la decomposizione del sistema nelle sue componenti \cite{harran2023design}. Evidenzieremo, inoltre, eventuali misure di sicurezza integrate nell'architettura e strategie impiegate per assicurare prestazioni efficienti.

\section{Riassunto del capitolo e conclusioni}

Alla fine di ogni capitolo includeremo un breve riassunto del suo contenuto, una riflessione su come quanto trattato contribuisca agli obiettivi generali della tesi e, per concludere, un'anticipazione di come i capitoli successivi faranno uso di quanto introdotto in quello corrente (in tal modo metteremo in evidenza come questi sono tra loro collegati) \cite{zobel2015writing}.